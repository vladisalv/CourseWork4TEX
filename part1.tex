\chapter{Формальная постановка задачи поиска повторов} \label{chapt1}

Молекула ДНК состоит из двух цепочек нуклеотидов, каждая из которых однозначно
определяется другой, что позволяет нам полностью кодировать молекулу ДНК одной 
из последовательностей. Последовательность состоит из конечного числа нуклеотидов,
обозначаемых A,C,G,T. Формально последовательность ДНК:

$$
X = (x_n)_{n=1}^N ,~ N \in \mathbb{N},~ x_n \in \{A,T,G,C\},
$$
где N - длина последовательности, A,T,G,C - озбоначения нуклеотидов.

\bigskip
Под повтором будем понимать пару последовательностей (X$_1$,X$_2$), для которых
справедливо неравенство:

$$
\rho(X_1,X_2) \le \epsilon,~|X_2|=|X_2|=K 
$$
где K - длина последовательностей, $\epsilon$ - константа,
а $\rho(X_1,X_2)$ - т.н. расстояние редактирования, оценка близости
последовательностей. Ее конкретный вид определяется алгоритмом.

\bigskip
Введем обозначение. Для последовательности X под X|$_i^k$ при условии $i+k<|X|$
будем понимать часть последовательности X с элемента с номером i длиной k:
$$
X|_i^k=\{x_i,x_{i+1},...,x_{i+k-1}\}
$$

\bigskip
Сформулируем в общем виде задачу поиска повторов в последовательностях ДНК.
Пусть есть две последовательности ДНК $X=(x_n)_{n=1}^{N_x}$ и $Y=(y_n)_{n=1}^{Nk_y}$.
Под задачей поиска повторов будем понимать нахождение всех троек $\{i_x,i_y,k\}
i_x,i_y,k \in \mathbb{N}$, таких что $i_x+k \le N_x, i_y+k \le N_y$ и
$(X|_{i_x}^k,Y|_{i_y}^k)$ - повтор длины k, т.е.:
$$
\rho(X|_{i_x}^k,Y|_{i_y}^k) \le \epsilon
$$

Одним из средств визуализации повторов найденных в последовательностях
являются точечные матрицы гомологии. Под матрицей гомологии с окном a и шагом s
для последовательностей $X=(x_n)_{n=1}^{N_x}$ и $Y=(y_n)_{n=1}^{Nk_y}$ будем
понимать
$$
M(X,Y,s,a)=(m_{ij})^{L_x*L_y}, ~L_x=[\frac{N_x-a}{s}], ~L_y=[\frac{N_y-a}{s}],
$$

$$
m_{ij}=
\left\{
  \begin{array}{rl}
    1,~$если$ X=(x_n)_{n=1}^{N_x} $и$ Y=(y_n)_{n=1}^{Nk_y} - $повтор$ \\
    0,~$иначе$ \\
  \end{array}
\right.
$$
\bigskip



\clearpage
