\chapter{Практическая реализация параллельного алгоритма} \label{chapt3}

Исходя из структуры алгоритма, его параллельная реализация вылилась в следующие
этапы вычислений:
\begin{enumerate}
  \item На первом шаге процессы считывают из исходной последовательности равные
  отрезки данных. Такое решение обусловлено возможностью выгрузки данных
  для дальнейшего использования и желанием сократить объем работы с файлом.
  Для разделенной работы с файлами использовались возможности стандарта MPI 2.0
  \item Обмен с соседями недостающими элементами. Разложение функций в спекторы.
  \item Сравнение полученных спекторов со спекторами других процессов.
  \item Построение гомологической матрицы
\end{enumerate}
Таким образом каждый этап представляет из себя переход данных из одного состояния
в последующее.
\bigskip
Для реализации разработанного алгоритма использовался язык программирования Си++
и технология интерфейса передачи сообщений MPI.
\bigskip
Основные функции классы реализации:

\begin{table}[htbp]
\caption{Классы, описывающие этапы реализации}
\label{tabular:timesandtenses}
\begin{center}
\begin{tabular}{| p{3cm} | p{10cm} |}
\hline
Класс & Описание  \\
\hline
Profile & Переводит текстовую последовательность в профили  \\
\hline
Decompose &  Раскладывает профили на спекторы. Обменивается недостоющими профилями
с соседними процессами\\
\hline
MatrixGomology &  Сравнивает спекторы друг с другом. Собирает спекторы других
процессов и формирует свою часть матрицы гомологии\\
\hline
Picture &  Собирает матрицу гомологии со всех процессоров. Переводит в изображение\\
MyMPI & Координирует работу всех процессов \\
\hline
\end{tabular}
\end{center}
\end{table}
\clearpage
