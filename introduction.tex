\chapter*{Введение}							% Заголовок
\addcontentsline{toc}{chapter}{Введение}	% Добавляем его в оглавление

Классификации живых организмов всегда была одной из основных задач биологии.
Наблюдения за животными и анатомический подход неизбежно приводили к ошибкам
в классификации, которые сейчас можно исправить благодаря развитию современных
технологий. Одной из таких технологий стала ДНК-ДНК гибридизация - биохимическая
реакция, протекающая с участием ДНК двух организмов для оценки их схожести.
Но, к сожалению, такая реакция очень дорогостоящая и недостаточно точная.

В последние десятилетия, мы можем наблюдать тенденцию замены экспериментов
их моделированием, что стало возможным используя вычислениям на ЭВМ. Тому есть
две причины: экономический фактор и возможность ускорения экспериментов.
Биоинформатика - наука, образовавшаяся на стыке молекулярной биологии, генетики,
математики и компьютерных технологий, основной целью которой является разработка
вычислительных алгоритмов для анализа и систематизации данных о структуре и
функциях биологических молекул, прежде всего нуклеиновых кислот и белков.
Одной из важных задач биоинформатики стала задача полногеномного сравнения ДНК
последовательностей и оценки их схожести.

Отдельного внимания заслуживает подзадача поиска повторов, т.к. повторы составляют
до 50\% нашего генома. Существует два вида повторов - разнесенные и тандемные.
Нас интересуют тандемные повторы - повторы, образовавшиеся в результате дублирования
фрагментов ДНК, когда копии фрагментов следуют точно друг за другом.
В зависимости от размера образца выделяют 4 вида тандемных повторов: мегасателиты
(длина свыше 1000 нуклеотидных пар), сателиты (от 100 до 1000 н.п.), минисателиты
(от 7 до 100 н.п.) и микросателиты (до 6 н.п.). Также, повторы различаются
по уровню сохранности образца. Дело в том, что со временем копированные образцы 
подвергаются мутациям, каждая копия дивергирует и тандемный повтор становится
размытым. Поэтому, выделяют три типа повторов по степени отличия от образца:
точные, несовершенные и размытые.

Практическая значимость тандемных повторов выходит из работ, доказывающих, что
тансоноспецифичность определенных повторяющихся участков генома позволяет
уточнить существующую классификацию организмов, используя геномные данные.

В настоящее время существуют алгоритмы, позволяющие успешно находить тандемные
повторы небольших размеров используя спектрально-статистические подходы.
Но до сих пор не существует специализированного метода для поиска мегасателитных
размытых повторов. А ведь именно мегасателитные повторы играют фундаментальную
роль в передаче по наследству генетических болезней и эволюции генома, что делает
задачу нахождения тандемных мегасателитных размытых повторов особенно актуальной.

Такое положение дел в последнее время наметилось из-за отставания
процесса вычислительного анализа генетических данных от стремительно развивающейся
экспериментальной базы в современной биологии. Т.к. геном можно представить
текстовой строкой из 4-х буквенного алфавита, то почти все алгоритмы обработки
генетических последовательностей были получены путем простой адаптации алгоритмов
обработки текстовой информации к условиям генетических текстов. В связи с этим,
временная сложность алгоритмов существенно нелинейна и при увеличении масштаба
сравнения и объёмов данных наблюдается резкое снижение эффективности таких
алгоритмов. Дело в том, что главным замедляющим фактором при сравнении схожих
генетических данных являются мутации, "исправление"\ которых существенно
увеличивает время анализа.

В данной работе рассматривается метод поиска тандемных протяженных размытых
повторов, разработанный коллективом разработчиков кафедры математических методов
прогнозирования и института математических проблем в Биологии Российской
академии наук (Пущино). Основное отличие данного метода состоит в переходе
от дискретного анализа к непрерывному анализу. Таким образом для
задачи дискретной по своей природе, применяется континуальный подход, основанный
на приближении непрерывных функций с помощью ортогональных многочленов.
Благодаря этому метод решает недостатки дискретного подохода и обладает
следующими свойствами:
\begin{itemize}
  \item Линейная сложность алгоритма
  \item Высокая степень устойчивости к мутациям
  \item Возможность распараллеливания алгоритма на кластерных системах
\end{itemize}

Целью курсовой работы является разработка параллельного алгоритма поиска
тандемных протяженных размытых повторов и анализ его эффективности. Исходя из
цели, можно выделить следующие задачи, поставленные в курсовой работе:
\begin{enumerate}
  \item Изучение предложенного метода поиска повторов.
  \item Разработка и реализация параллельного алгоритма.
  \item Исследование масштабируемости алгоритма.
\end{enumerate}


\cite{bib1}

\clearpage
